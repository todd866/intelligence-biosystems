\documentclass[11pt]{letter}
\usepackage[margin=1in]{geometry}
\usepackage{hyperref}
\usepackage{amsmath}

\signature{Ian Todd\\Sydney Medical School\\University of Sydney}
\address{Ian Todd\\Sydney Medical School\\University of Sydney\\Sydney, NSW, Australia\\itod2305@uni.sydney.edu.au}

\begin{document}

\begin{letter}{Dr. Abir Igamberdiev\\Editor-in-Chief\\BioSystems}

\opening{Dear Dr. Igamberdiev,}

I am pleased to resubmit my revised manuscript titled ``Intelligence as High-Dimensional Coherence: The Observable Dimensionality Bound and Computational Tractability'' (BIOSYS-D-25-00880R1) for reconsideration in \textit{BioSystems}.

Thank you for the constructive feedback on the previous revision. I agree with your assessment that the framing should emphasize that intelligence \textit{arises from} the maintenance of high-dimensional coherence, rather than treating intelligence as identical to dimensionality. As you correctly note, intelligence is not a physical property like time and space---it is an emergent capacity that depends on underlying physical processes.

This reframing actually strengthens the theoretical argument: the claim is not that intelligence \textit{is} high-dimensional (an ontological identity), but that intelligent behavior \textit{arises from} systems that maintain high-dimensional coherent dynamics (a causal/mechanistic claim). This distinction is philosophically important and I appreciate your drawing attention to it.

\textbf{Summary of Changes:}

\begin{enumerate}
\item \textbf{Abstract reformatted}: Now presented as a single paragraph of approximately 240 words, without bold formatting, with self-explanatory statements that do not rely on undefined symbols.

\item \textbf{Core claim reframed}: Changed ``Intelligence must be high-dimensional'' to ``Intelligence arises from the maintenance of high-dimensional coherent dynamics'' throughout the manuscript, including abstract, introduction, and conclusion.

\item \textbf{Recent BioSystems literature integrated}: Added citations to three recent papers on consciousness and dimensionality (Gunji et al.\ 2020, Igamberdiev 2024, Gunji 2025), discussed in Section 8.3 (Relationship to Existing Frameworks) with explicit connections to the measurement-theoretic framework.

\item \textbf{Technical corrections}: Added explicit $D_{\text{target}}$ definition; corrected Table 2 collision counts (continuous = 1 at readout, not 0).

\item \textbf{Quantum section clarified}: The discussion of structural parallels with Igamberdiev (1993) now explicitly states that biology looks ``quantum'' because high-dimensional dynamics produce quantum-like phenomenology---not because biology \textit{is} quantum mechanical. Both frameworks ``rhyme'' because they share the same upstream cause (high dimensionality with limited measurement bandwidth).
\end{enumerate}

The core theoretical framework, quantitative demonstrations, and empirical predictions remain unchanged from the R1 revision. This R2 revision addresses the editorial comments on framing and formatting while preserving the substantive contributions that reviewers found valuable.

A point-by-point response to your editorial comments is provided separately.

Thank you again for your guidance in strengthening this manuscript.

\closing{Sincerely,}

\end{letter}

\end{document}
