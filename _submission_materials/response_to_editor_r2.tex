\documentclass[11pt]{article}
\usepackage[margin=1in]{geometry}
\usepackage{xcolor}
\usepackage{hyperref}
\usepackage{amsmath}

\definecolor{editorcolor}{rgb}{0.0, 0.0, 0.6}
\definecolor{responsecolor}{rgb}{0.0, 0.4, 0.0}

\newcommand{\editor}[1]{\textcolor{editorcolor}{\textbf{#1}}}
\newcommand{\response}{\textcolor{responsecolor}{\textbf{Response: }}}

\title{\vspace{-1cm}Response to Editor\\
\large Manuscript BIOSYS-D-25-00880R1\\
``Intelligence as High-Dimensional Coherence: The Observable Dimensionality Bound and Computational Tractability''}
\author{Ian Todd}
\date{\today}

\begin{document}

\maketitle

I thank the Editor-in-Chief for the constructive feedback on the R1 revision. Below I provide detailed responses to each editorial comment.

\section*{Editorial Comments and Responses}

\editor{Comment 1:} It is more relevant to say that intelligence arises from the maintenance of high-dimensional coherence rather than that intelligence itself is high-dimensional, as intelligence is not a physical property like time and space.

\response I agree entirely with this distinction. Intelligence is an emergent capacity, not a physical property, and the original phrasing conflated mechanism with identity.

\textbf{Changes made:}
\begin{itemize}
\item \textbf{Abstract}: Changed opening from ``Intelligence must be high-dimensional'' to ``Intelligence arises from the maintenance of high-dimensional coherent dynamics.''

\item \textbf{Section 1.1 (The Core Claim)}: Changed ``Intelligence must be high-dimensional because tracking complexity is what intelligence means'' to ``Intelligence arises from maintaining high-dimensional coherence because tracking complexity is what intelligence means.''

\item \textbf{Section 1.1}: Changed ``Intelligence is implemented through high-dimensional continuous dynamics'' to ``Intelligent behavior arises from high-dimensional continuous dynamics.''

\item \textbf{Conclusion}: Changed ``We establish intelligence as the capacity to maintain high-dimensional coherent dynamics'' to ``We establish that intelligence arises from the capacity to maintain high-dimensional coherent dynamics.''
\end{itemize}

This reframing strengthens the argument: the claim is now explicitly causal/mechanistic (``arises from'') rather than ontological (``is''), which is philosophically more defensible and scientifically more precise.

\vspace{0.5cm}

\editor{Comment 2:} The abstract should be organized in one paragraph of no more than 250 words and be self-explanatory. Currently, it contains the statements that are at least partially explained in the main text but remain obscure in the abstract, e.g. regarding the values of $D_{\text{target}}$, the values of energy dissipation, etc. The statement is presented in bold font in the abstract, which does not correspond to the rules of formatting abstracts.

\response The abstract has been completely rewritten to address all formatting concerns.

\textbf{Changes made:}
\begin{itemize}
\item \textbf{Single paragraph}: The abstract is now one continuous paragraph (previously three).

\item \textbf{Word count}: Approximately 240 words (within the 250-word limit).

\item \textbf{No bold formatting}: Removed all bold text from the abstract.

\item \textbf{Self-explanatory}: Removed specific numerical values ($D_{\text{target}} \sim 10^1$--$10^2$, energy dissipation values, etc.) that required context from the main text. The abstract now describes concepts in accessible language without assuming familiarity with the notation.

\item \textbf{Logical flow}: The abstract now follows a clear progression: (1) main claim, (2) core constraint, (3) how biology sidesteps the constraint, (4) implications for efficiency, (5) formal contributions, (6) connection to existing literature.
\end{itemize}

The revised abstract can be understood independently without reference to the main text.

\vspace{0.5cm}

\editor{Comment 3:} Regarding the dimensionality of consciousness, it may be useful to consider the recently published papers in BioSystems:
\begin{itemize}
\item \url{https://doi.org/10.1016/j.biosystems.2025.105677}
\item \url{https://doi.org/10.1016/j.biosystems.2024.105346}
\item \url{https://doi.org/10.1016/j.biosystems.2020.104151}
\end{itemize}

\response I have integrated these three papers into the manuscript, situating the current work within the broader \textit{BioSystems} literature on consciousness and dimensionality.

\textbf{Changes made:}
\begin{itemize}
\item \textbf{Section 8.3 (Relationship to Existing Frameworks)}: Added a dedicated itemized list citing all three papers with explicit connections to our framework:

\begin{itemize}
\item \textbf{Gunji et al.\ (2020)}: Free will, determinism, and locality form a trilemma. Their ``Type III'' structure (no locality) enables free will and determinism to coexist because internal contexts become isolated from external observation---directly paralleling our claim that $D_{\text{eff}} > D_{\text{crit}}$ makes internal dynamics observationally inaccessible.

\item \textbf{Igamberdiev (2024)}: Three hierarchical thermodynamic cycles where each cycle detects low-energy quanta from the previous; internal dynamics involve ``non-demolition quantum measurements'' with extremely low dissipation. This directly parallels our claim that Landauer cost concentrates at behavioral outputs.

\item \textbf{Gunji (2025)}: Distinguished ``Natural Born Intelligence'' from AI---AI treats the unknown as not-yet-experienced data within a fixed representational space, while living systems access a ``true outside.'' The correlationist assumption implicitly requires $D_{\text{target}} < D_{\text{crit}}$; Gunji's ``true outside'' corresponds to dynamics beyond the observable dimensionality bound.
\end{itemize}

\item \textbf{Abstract}: Added closing sentence connecting to ``recent work on consciousness and dimensionality.''

\item \textbf{References}: Added complete bibliographic entries for all three papers with DOIs.
\end{itemize}

These citations strengthen the paper's connection to ongoing work in \textit{BioSystems} and demonstrate how the current framework contributes to the journal's broader discourse on consciousness, complexity, and biological organization.

\section*{Summary of R2 Changes}

\begin{enumerate}
\item \textbf{Philosophical reframing}: ``Intelligence IS high-dimensional'' $\to$ ``Intelligence ARISES FROM high-dimensional coherence'' throughout manuscript.

\item \textbf{Abstract reformatted}: Single paragraph, $\sim$240 words, no bold, self-explanatory without undefined notation.

\item \textbf{BioSystems literature integrated}: Three recent papers on consciousness/dimensionality cited and discussed in context of existing theoretical frameworks.

\item \textbf{Definition clarity}: Added explicit definition of $D_{\text{target}}$ (causal target dimensionality) in Section 1.1 to prevent notation drift.

\item \textbf{Collision counting corrected}: Table 2 and associated text now correctly show continuous collisions = 1 (the final readout), not 0. This reflects that Landauer cost is paid once at measurement, making the cost ratio well-defined.

\item \textbf{Quantum section reframed}: The discussion of Igamberdiev (1993) now emphasizes that biology looks ``quantum'' because high-dimensional dynamics with limited measurement bandwidth produce quantum-like phenomenology---not because biology \textit{is} quantum mechanical. The framing is now explicitly: ``They rhyme because they share the same upstream cause'' (high dimensionality), avoiding any claim that classical high-D physics \textit{is} quantum mechanics.
\end{enumerate}

No changes were made to the core theoretical framework, quantitative demonstrations, or empirical predictions from the R1 revision. The R2 changes address editorial guidance on framing and formatting while preserving the substantive contributions.

I thank the editor for the thoughtful feedback, which has improved both the precision and accessibility of the manuscript.

\end{document}
